\newpage
\section{Bibliographie}
\subsection{Courbes elliptiques}
Une courbe elliptique est l'ensemble $$ E = \{ (x,y) \in \mathbb{R} | y^2 = x^3 +ax +b\} $$avec a,b réels tels que $4a^3 + 27 b^2 \ne 0$\\
 - La courbe est symétrique avec l'axe des abscisse (selon x)\\
 - La condition sur a,b est le discriminant de l'équation du troisième degré\\
 - Presque toute les droites coupant par deux points la courbe passent par un troisième point (sauf droite à x constant)\\

![[exempleCourbe.png]]
Exemple ici avec a=-3, b=1

On a besoin d'une loi de composition interne, notée + ($E \times E \to E)$, pour former un groupe $(E,+)$\\

Rappel de la définition d'un groupe:\\
 - la loi est associative : pour tout a,b,c de E, a+(b+c) = (a+b)+c\\
 - E possède un élément neutre\\
 - Les éléments de E possède un inverse dans E pour + (il existe un élément y dans E tel que x+y vaut l'élément neutre)\\


On définit + tel que, pour tout point P,Q de E\\
- Si P et Q sont symétriques, ils sont symétriques selon l'axe des abscisses, on a $(x_q,y_q) = (x_p,-y_p)$, alors $P+Q = N_e$,  point neutre , on note $P= -Q$\\
 - En général: La droite formée par P et Q coupe la courbe en un troisième point R, on a $P+Q = - R$\\
 - Si P = Q, la droite est la tangente de la courbe en P, on note $P+P = 2P$\\

Difficulté du problème : Pour tout entier naturel n, calculer $S = nP$ est facile, mais retrouver n avec S et P est très difficile\\

\subsection{Application au cryptage}
On (Bob) choisit une clé privée n et un point P. \\
La clé publique est :\\
 - $Q = nP$\\
 - P\\
 - la courbe (donc a et b)\\

Alice veut envoyer un message M. Elle choisit k entier > 1, et envoie à Bob les deux points $kP$ et $M+kQ$ \\

Bob connait n donc retrouve $M = (M+kQ) - nkP$\\


Le cryptage sur courbe elliptiques est un cryptage asymétrique : il y a une clé privée et une clé publique : Trouver la clé privée est difficile, vérifier la clé est facile. 
En théorie, on découpe le message à chiffrer en petits blocs qui sont chacun chiffré.
En pratique on utilise le cryptage asymétrique pour chiffrer la clé d'un cryptage symétrique comme AES.
